\documentclass[UTF8]{ctexart}
\usepackage{graphicx}
\usepackage{listings}
\usepackage[colorlinks,linkcolor=red]{hyperref} %把颜色改成红色
\usepackage{xcolor}
\usepackage{amsmath}
\newtheorem{corollary}{Corollary}[section]

\title{lca与rmq模型问题整理}
\author{赵涵铮}
\date{\today}	

\lstset{
	basicstyle          =   \sffamily,          % 基本代码风格
	keywordstyle        =   \bfseries,          % 关键字风格
	commentstyle        =   \rmfamily\itshape,  % 注释的风格,斜体
	stringstyle         =   \ttfamily,  % 字符串风格
	flexiblecolumns,                % 别问为什么,加上这个
	numbers             =   left,   % 行号的位置在左边
	showspaces          =   false,  % 是否显示空格,显示了有点乱,所以不现实了
	numberstyle         =   \zihao{-5}\ttfamily,    % 行号的样式,小五号,tt等宽字体
	showstringspaces    =   false,
	captionpos          =   t,      % 这段代码的名字所呈现的位置,t指的是top上面
	frame               =   lrtb,   % 显示边框
}

\lstdefinestyle{cpp}{
	language        =   C++, % 语言选Python
	basicstyle      =   \zihao{-5}\ttfamily,
	numberstyle     =   \zihao{-5}\ttfamily,
	breaklines      =   true,   % 自动换行,建议不要写太长的行
	columns         =   fixed,  % 如果不加这一句,字间距就不固定,很丑,必须加
	basewidth       =   0.5em,
}

\begin{document}
	\maketitle
	
	\section{引言}
	在第一次课中,我们浅谈了lca与rmq问题,并实现了其模板题\href{https://www.luogu.com.cn/problem/P3379}{P3379 LCA}和\href{https://www.luogu.com.cn/problem/P3865}{P3865 st表},并留下了\href{https://www.mfstem.org/homework/63e64885b2e20e03a62838e3}{沐枫网上的六道练习题}作为作业。接下来将会进行逐一的讲解。
	
	\section{数列区间最大值}
	\subsection{题目描述}
	输入数组$a$,给定 $M$ 个询问,每次询问给定 $X$, $Y$,要求输出 $X$ 到 $Y$ 这段区间内的最大数。
	\subsection{数据范围}
	$1\leq N\leq 10^6,1\leq M\leq 10^6,1\leq X \leq Y\leq N$,数字不超过 C/C++ 的 int 范围。
	\subsection{解题思路}
	这道题目是典型的RMQ问题,我们依然考虑倍增去预处理数组,使得对于任意数字$i \in [1,n]$,我们都能快速求出来$[i,i+2^j],j\in (0,20)$区间内的最大值。我们令$Max[i][j]$表示从$a_i$开始,$[i,i+2^j]$的区间内的最大值,考虑状态转移,可以通过:
	$$
	Max[i][j] = max(Max[i][j-1],Max[i+2^j][j-1])
	$$
	得到,可以这么理解:一个大区间可以分成相同大小的两个小区间,大区间的最大值就是两个小区间中较大的值。
	\subsection{参考代码}
	\lstinputlisting[
	style       =   cpp,
	caption     =   {数列区间最大值参考代码},
	label       =   {problem1.cpp}
	]{problem1.cpp}
	
	\section{天才的记忆}
	这个问题同第一题“数列区间最大值”。
	\subsection{参考代码}
	\lstinputlisting[
	style       =   cpp,
	caption     =   {数列区间最大值参考代码},
	label       =   {problem1.cpp}
	]{problem1.cpp}
\end{document}